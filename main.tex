\font\jhhei="Microsoft JhengHei"
\font\Title="Heiti TC" at 18pt
\font\Section="Heiti TC" at 14pt
\font\Subsection="Heiti TC" at 12pt
\font\xesl="FreeSerif/I"
\font\ComputerModernTypewritter="[/usr/share/fonts/X11/Type1/Type1/sftt1000.pfb]"

\newcommand{\xett}{\ComputerModernTypewritter}
\newcommand\titlename{数字图像处理第六次作业}
\newcommand\vicetitle{}
\newcommand{\myauthor}{李超凡  091180066}

\documentclass[a4paper,10pt]{article}
\usepackage[top=1in, bottom=1.2in, left=1.25in, right=1.25in]{geometry}
\usepackage{url}
\usepackage{titlesec} 
\usepackage{natbib} 
\usepackage{amsthm,amsmath,yhmath} 
\usepackage{graphicx,caption2} 
\usepackage{amssymb} 
\usepackage{indentfirst} 
\usepackage{pdfpages}		% 插入pdf
\usepackage{fontspec}
\usepackage{xunicode}
\usepackage{enumerate}
\usepackage{listings}
%for table
\usepackage{booktabs}
\usepackage{multirow}
%%%%%%%%%%%%%%%%%%
\usepackage{pst-circ}
%%%%%%%%%%%%%%%%%%%%%%%%%%%
\usepackage{color}
\usepackage{xcolor}

\usepackage[colorlinks,linkcolor=blue,citecolor=blue,
		pdfauthor={Fanchy Lee},
		pdftitle={\titlename},
		pdfkeywords={},
		pdfproducer={XeLateX with hyperref},
		pdfcreator={Xelatex}]{hyperref}		% 让tableofcontents支持超链接

\XeTeXlinebreaklocale="zh"	% 设置中文换行
\XeTeXlinebreakskip=0pt plus 1pt minus 0.1pt
%\setlength{\parindent}{2em}	% 首行缩进,2字符
%\allowdisplaybreaks

%fonts
\setromanfont[Mapping=tex-text]{MyPMingLiU}
\setsansfont[Mapping=tex-text]{Heiti TC}
%\setmonofont{Courier}

\lstset{
	language=C,
	basicstyle=\footnotesize\ttfamily, % Standardschrift
	numbers=left,               % Ort der Zeilennummern
	numberstyle=\tiny,          % Stil der Zeilennummern
	stepnumber=1,              % Abstand zwischen den Zeilennummern
	numbersep=5pt,              % Abstand der Nummern zum Text
	tabsize=4,                  % Groesse von Tabs
	extendedchars=true,         %
	breaklines=true,            % Zeilen werden Umgebrochen
	keywordstyle=\color{blue},
	frame=br,         
	captionpos= b,
%	keywordstyle=[1]\textbf,    % Stil der Keywords
%	keywordstyle=[2]\textbf,    %
%	keywordstyle=[3]\textbf,    %
%	keywordstyle=[4]\textbf,   \sqrt{\sqrt{}} %
	stringstyle=\color{red}\ttfamily, % Farbe der String
	showspaces=false,           % Leerzeichen anzeigen ?
	showtabs=false,             % Tabs anzeigen ?  xleftmargin=17pt, framexleftmargin=17pt, framexrightmargin=5pt, framexbottommargin=4pt, %backgroundcolor=\color{lightgray},
	showstringspaces=false,     % Leerzeichen in Strings anzeigen ?        
}


\renewcommand{\abstractname}{摘\  要}
\renewcommand{\refname}{参考文献}			% 参考文献
\renewcommand{\tablename}{表}
\renewcommand{\figurename}{图}
\renewcommand{\captionlabeldelim}{. }
\renewcommand{\today}{{\number\year -\number\month -\number\day }}
\renewcommand{\lstlistingname}{代码}


\titleformat{\section}[hang]{\Section }{\thesection}{3pt}{ }
\titleformat{\subsection}[hang]{\Subsection }{\thesubsection}{3pt}{}

\theoremstyle{definition}
\newtheorem{defn}{定义}
\theoremstyle{plain}
\newtheorem{lemma}{引理}

%\input{noauthortitle.tex}
%空白页
\makeatletter
\def\cleardoublepage{
    \clearpage\if@twoside\ifodd\c@page\else
    \hbox{}
    \vspace*{\fill}
    \begin{center}
	%anything you want to add ? just add here
    \end{center}
    \vspace{\fill}
    \thispagestyle{empty}
    \newpage
    \if@twocolumn\hbox{}\newpage\fi\fi\fi
}
\makeatother

\begin{document}
\title{\Title \titlename\\  \vicetitle}
\author{ 南京大学电子科学与工程学院\\   \myauthor} 
\maketitle

\section{图像水平平移叠加后的退化函数}

以水平平移长度为 $a$ 后与原始图像叠加后造成的图像退化({\xesl degradation})为例,假定原始图像可以用 $f(x,y)$ \footnote{使用连续函数模型,而不是实际数字图像所采用的离散模型。}来表示,那么水平平移 $a$ 后的图形可以用 $f(x-a,y)$ 来表示。这两个函数的傅利叶变换分别可以表示为:
\begin{align*}
f(x,y)&\leftrightarrow F(u,v)\\
f(x-a,y)&\leftrightarrow F(u,v)e^{-j2\pi au}
\end{align*}
于是退化后的图像函数 $g(x,y) = f(x,y)+f(x-a,y)$ 的傅利叶变换为:
\begin{align*}
G(u,v)& = F(u,v)+F(u,v)e^{-j2\pi au}\\
&=(1+e^{-j2\pi au})F(u,v)
\end{align*}
而如果退化函数用 $h(x,y)$ 来表示,则有 $g(x,y) = (h*f)(x,y)$ \footnote{忽略噪声,$(h*f)$ 表示两个函数的卷积。},那么 $h(x,y)$ 的傅利叶变换显然为:
\begin{align*}
H(u,v)&=\frac{G(u,v)}{F(u,v)}\\
&=1+e^{-j2\pi au}
\end{align*}
\section{维纳滤波复原}
忽略噪声的图像退化可以简化为原始图像函数与退化函数卷积,退化图像的复原({\xesl Restoration})即是解卷积({\xesl deconvolution})的过程,这个过程在傅利叶变换域即为拿退化后的 $G(u,v)$ 乘以 $\frac{1}{H(u,v)}$。

维纳滤波为这一过程的改进形式,它考虑噪声的干扰。具体形式如下:
\[
V(u,v)=\frac{1}{H(u,v)}\frac{|H(u,v)|^2}{|H(u,v)|^2+S_\eta(u,v)/S_f(u,v)}
\]
其中的 $S_\eta(u,v)$ 和 $S_f(u,v)$ 分别代表噪声的功率谱和原始图像的功率谱,通常情况下,$S_\eta(u,v)/S_f(u,v)$ 用一个常数 $K$ 代替。在没有噪声的情况下,$K=0$ 维纳滤波退化为 $1/H(u,v)$ 。维纳滤波的代码如下:
\lstinputlisting[caption=wiener.m,language=matlab]{wiener.m}
其中的 {\tt G}、{\tt H} 分别为退化后图像的傅利叶变换,和退化函数的傅利叶变换。

由于没有加入噪声,取参数 {\tt K} 为 0,滤波结果如图 \ref{wiener} 所示:
\begin{figure}[htbp]
\centering
\includegraphics[scale=.8]{Lenna_wiener.jpg}
\caption{维纳滤波前后对比\label{wiener}}
\end{figure}
\end{document}

